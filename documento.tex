\documentclass[10pt]{article}

\usepackage[utf8]{inputenc}
\usepackage[italian]{babel}
\usepackage{url}
\usepackage{graphicx}
\usepackage[a4paper,centering,top=0.2cm,bottom=1.3cm,outer=0.5cm,inner=0.5cm]{geometry}

\title{\small{Diventa Data Scientist!}}
\author{\small{Filippo Valle}\\ \texttt{\href{mailto:mail@fvalle.me?Subject=Diventa\%20Data\%20Scientist}{mail@fvalle.me}}}
\date{\small{\today}}

\usepackage{hyperref}	%indice pdf

\hypersetup{pdfauthor={Filippo Valle}, pdftitle={Gara: Diventa Data Scientist},
 pdfproducer={\LaTeX},
 bookmarks=true,
 unicode=true,
 pdftoolbar=false,
 pdfmenubar=false,
 pdffitwindow=false, 
 pdfstartview=FitH,
colorlinks=true,
urlcolor=cyan,
linkcolor=magenta
}


\begin{document}
\maketitle
\section{Tools}
\label{sec:tools}
Per sviluppare il problema ho \textbf{scrito un codice nel linguaggio R}, ed ho fatto uso dei pacchetti\footnote{\url{https://cran.r-project.org/web/packages/}}: \emph{rattle}, \emph{rpart.plot}, \emph{randomforest}.

\section{Organizzazione dei dati}
\subsection{Pulizia}
\label{sec:pulizia}
Per togliere i dati ``\emph{sporchi}'' dai set di dati ho  \textbf{eliminato} da \emph{train} le righe:
\begin{itemize}
\item con date multiple
\item con le colonne allineate in maniera errata  
\end{itemize}

\subsection{Dati mancanti}
\label{sec:datimancanti}
ho colmato i dati mancanti:
\begin{itemize}
\item per la variabile \textbf{Punteggio} ho assegnato la mediana ai dati \emph{NA} e a quelli \emph{equipment or floor not properly drained} 
\item ho assegnato il \textbf{Coefficiente di livello} in base alla tabella \ref{tab:punteggi}
\item ho considerato nel valore \textbf{Criticità}, \emph{anti-siphonage or backflow prevention device not provided where required} come \emph{Critical}, in modo da avere gli stessi valori (\emph{Critical}, \emph{Not Critical}, \emph{Not Applicable})  in entrambi i set di dati
\item ho cercato con un \textbf{decision tree} il \textbf{Quartiere} per i dati in cui risultava \emph{Missing}; per cercare il quartiere ho usato come parametro principale il Codice postale (figura \ref{fig:quartieretree})
\end{itemize}

\subsection{Nuove categorie}
\label{sec:nuovecategorie}
Ho dovuto\footnote{Con più di 54 categorie rpart da' errore} raggruppare i valori in macro aree, ho altresì aggiunto nuove variabili al modello:
\begin{itemize}
\item ho raggruppato \textbf{Tipo cucina}, per esempio la categoria \emph{Californian} e la categoria \emph{Hawaiian} sono raggruppate nella macro-categoria \emph{American}
\item ho aggiustato \textbf{Tipo ispezione}, per esempio, considerando cosiderando \emph{re-opening} come \emph{initial} trattando quindi che un locale che ha riaperto come un locale nuovo
\item ho creato delle \textbf{Zone} basandomi sul Codice postale
\item ho creato una variabile \textbf{Prefisso telefonico} poiché a New York può essere importante\footnote{\url{http://archive.is/2016.08.07-072128/http://www.linkiesta.it/it/article/2011/04/26/a-new-york-i-prefissi-telefonici-li-mettono-allasta/618/}}!
\end{itemize}
\begin{table}[p!]
\centering
\begin{tabular}{| c | c |}
    \hline
    Regola & Coefficiente\\ \hline
\(p\leq13\)&A\\ \hline
\(13<p\leq27\)&B \\ \hline
\(27<p\)&C\\ \hline
\end{tabular}
\caption{Dal sito della gara (\emph{p} rappresenta il punteggio ottenuto)}
\label{tab:punteggi}
\end{table}

\begin{figure}[p!]
\centering
\includegraphics[width=0.65\linewidth]{p_quartiere.pdf}
\caption{Albero per decidere il quartiere}
\label{fig:quartieretree}
\end{figure}

\section{Random Forest}
\label{sec:randomforest}
Per cercare \textbf{quale Azione sarebbe stata intrapresa nei confronti del ristorante} ho usato un algoritmo \textbf{random forest}.

Ho provato ad includere quanti più parametri possibili e ho capito, come riportato nel  diagramma in figura \ref{fig:random}, che varibili come \textbf{Criticità}, \textbf{Tipo ispezione} e \textbf{Punteggio} hanno un peso importante!
\begin{figure}[p!]
\centering
\includegraphics[width=0.45\linewidth]{forest_graph.pdf}
\caption{importanza delle diverse variabili nella costruzione della random forest (più \emph{MeanDecreaseAccuracy} è alto più la varibile ha importanza nella simulazione)}
\label{fig:random}
\end{figure}
Dopo vari tentativi volti ad ottimizzare l'uso delle varibili ho costruito la \textbf{foresta principale}, da cui ho poi creato la tabella dei risultati.

\section{Risultati}
\label{sec:risultati}
Prima di compilare i risultati ho supposto che nel caso:
\begin{itemize}
\item \emph{Establishment re-closed by DOHMH} il locale \textbf{chiudesse}
\item \emph{Establishment re-opened by DOHMH} il locale \textbf{non avesse violazioni}
\end{itemize} 

Ho infine salvato i risultati nel file \emph{submit.csv}.
\end{document}
